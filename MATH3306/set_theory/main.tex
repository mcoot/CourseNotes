\documentclass{article}

%%%%%%%%%%%%%%%%%%%%%%%%%%%%%%%%%%%%%%%%
%% Packages
%%%%%%%%%%%%%%%%%%%%%%%%%%%%%%%%%%%%%%%%

\usepackage[english]{babel}

% Margins
\usepackage[margin=1in]{geometry}
% Spaced paragraphs
\usepackage[parfill]{parskip}
% Fancy headers
\usepackage{fancyhdr}
% Allow use of titles
\usepackage{titling}

% Maths stuff
\usepackage{amsmath}
\usepackage{amsfonts}
\usepackage{amssymb}
\usepackage{amsthm}
\usepackage{bm}
\usepackage{turnstile}

% Links
\usepackage[hidelinks]{hyperref}

% Allow repetition in macros
\usepackage{multido}

%%%%%%%%%%%%%%%%%%%%%%%%%%%%%%%%%%%%%%%%
%% Set Theory Macros
%%%%%%%%%%%%%%%%%%%%%%%%%%%%%%%%%%%%%%%%

% Use small arrows for implies / iff
\renewcommand{\implies}{\rightarrow}
\renewcommand{\iff}{\leftrightarrow}

% Notation for a set (easier than \lbrace \rbrace :P)
\newcommand{\set}[1]{\left\lbrace #1 \right\rbrace}

% Notation for a set comprehension / set builder notation
\newcommand{\setcomp}[2]{\set{#1 \, | \, #2}}

% Notation for a set comprehension with two conditions
\newcommand{\setcompdouble}[3]{\set{#1 \, | \, #2 \, | \, #3}}

% Notation for a tuple
\newcommand{\tuple}[1]{\left\langle #1 \right\rangle}

% Notation for a pair
\newcommand{\pair}[2]{\tuple{#1, #2}}

% Powerset
\newcommand{\powerset}[1]{\mathcal{P}(#1)}

% Ordinals
\newcommand{\On}{\bm{On}}
% Language
\newcommand{\Lang}{\mathcal{L}}
% Language of Set Theory
\newcommand{\LangSet}{\Lang_\in}

% Model
\newcommand{\Model}{\mathcal{M}}

% Function notation for injection / surjection / bijection
\newcommand{\surject}{\twoheadrightarrow}
\newcommand{\inject}{\rightarrowtail}
\newcommand{\biject}{\mathbin{\rightarrowtail \hspace{-8pt} \twoheadrightarrow}}

%%%%%%%%%%%%%%%%%%%%%%%%%%%%%%%%%%%%%%%%
%% Titles
%%%%%%%%%%%%%%%%%%%%%%%%%%%%%%%%%%%%%%%%

\title{MATH3306 Course Notes -- \url{https://github.com/mcoot/CourseNotes} }
\author{Joseph Spearritt}

\pagestyle{fancy}
\lhead{\thetitle}
\rhead{\theauthor}
\cfoot{\thepage}

\begin{document}
	\section{Graph Model of Set Theory}

\begin{itemize}
	
	\item Directed graphs: $ G = \pair{V}{A} $
	
%	\begin{itemize}
%		
%		\item $ V $ is the set of vertices
%		
%		\item $ A \subseteq V \times V $ is the set of ordered pairs of vertices representing arrows
%		
%	\end{itemize}

	\item A graph is \textbf{well-founded} if it has no looping paths and no infinite descending paths
	
	\item A graph is \textbf{extensional} if for any $ v_0, v_1 $ such that $ v_0 $ has the same incoming arrows as $ v_1 $, $ v_0 = v_1 $
	
	\item Two graphs are isomorphic if there is a function (\textit{isomorphism}) $ \sigma $ between them such that:
	
	\begin{itemize}
		
		\item $ \sigma $ is a bijection (surjection + injection)
		
		\item $ vA_0 u \leftrightarrow \sigma(v) A_1 \sigma(u) $
		
	\end{itemize}

	\item An automorphism is an isomorphism between some graph and itself (the identity is a trivial one)
	
	\item $ G $ is a subgraph of $ G' $ if $ V \subseteq V' $ and $ v_0 A v_1 \leftrightarrow v_0 A' v_1 $ for all $ v_0, v_1 \in V $
	
	\item $ G $ is \textit{maximal} in some property $ \Phi $ if $ G $ possesses $ \Phi $ and there exists no graph $ G' $ such that:
	
	\begin{itemize}
		
		\item $ G' $ possesses $ \Phi $; and
		\item $ G $ is a proper subgraph of $ G' $
		
	\end{itemize}
	
	\item Let $ G $ be a \textit{maximal} well-founded graph with no non-trivial automorphisms
	
	\item Equivalently, $ G $ is a maximal well-founded graph which is extensional
	
	\item $ G $ is then an \textit{intended model} of Set Theory
	
\end{itemize}

\section{First Order Logic}

\begin{itemize}
	
	\item Logical symbols: $ \lnot $, $ \land $, $ \lor $, $ \implies $, $ \iff $, $ \exists $, $ \forall $, $ ( $, $ ) $
	
	\item Non-logical symbols: constant symbols ($ a, b, c $), relation symbols ($ P, Q, R $), function symbols ($ f, g, h $)
	
	\item Language: $ \Lang = \set{a, b, c, \dots, P, Q, R, \dots, f, g, h, \dots}$
	
	\item Individual variables will be denoted $ v_1, v_2, \dots $
	
	\item Metavariables / arbitrary variables will be denoted $ x, y, z, \dots $
	
	\item The set of terms, $ Term $ is defined recursively:
	
	\begin{itemize}
		
		\item If $ t $ is a constant symbol or individual variable, $ t $ is a term
		
		\item If $ t_1, \dots, t_n $ are terms and $ f $ is a function symbol with arity $ n $, $ f(t_1, \dots, t_n) $ is a term
		
		\item Nothing else is a term
		
	\end{itemize}
	
	\item A string $ \varphi $ is an \textit{atom} if $ \varphi = Rt_1, \dots, t_n $ where $ t_1, \dots, t_n $ are terms and $ R $ is a relation with arity $ n $
	
	\item If $ t $ is a term with no variables occurring in it, $ t $ is a \textit{closed term}
	
	\item The set of Well Formed Formulae $ WFF $ is defined recursively, with atoms as the base case
	
	\begin{itemize}
		\item If $ \varphi $ consists of a combination of logical operators over well formed formulae, $ \varphi \in WFF $
	\end{itemize}
	
%	\item The set of Well Formed Formulae, $ WFF $ is defined recursively:
%	
%	\begin{itemize}
%		
%		\item If $ \varphi $ is an atom, $ \varphi \in WFF $
%		
%		\item If $ \varphi \in \set{\lnot \psi, \forall x \psi, \exists x \psi} $ where $ \psi \in WFF $ and $ x $ is a variable, $ \varphi \in WFF $
%		
%		\item If $ \varphi \in \set{(\psi \land \chi), (\psi \lor \chi), (\psi \implies \chi)} $ where $ \psi, \chi \in WFF $, $ \varphi \in WFF $
%		
%		\item Nothing else is in $ WFF $
%		
%	\end{itemize}

	\item $ x $ is \textit{free} in $ \varphi \in WFF $ if:
	
	\begin{itemize}
		
		\item If $ \varphi = R t_1, \dots, t_n $ and $ x = t_i $ for some $ i $
		
		\item If $ \varphi $ consists of logical operators over well formed formula, where $ x $ is free in at least one
		
		\item If $ \varphi $ is a quantification over a formula where $ x $ is free, and $ x $ is not the bound variable
		
	\end{itemize}

%	\item The variable $ x $ `being free' in $ \varphi \in WFF $ is defined recursively:
%	
%	\begin{itemize}
%		
%		\item If $ \varphi = R t_1, \dots, t_n $ and $ x = t_i $ for some $ i $, then $ x $ is free in $ \varphi $
%		
%		\item If $ \varphi \in \set{\forall y \psi, \exists y \psi} $ and $ x $ is free in $ \psi $ and $ x \ne y $, $ x $ is free in $ \varphi $
%		
%		\item If $ \varphi \in \set{\lnot \psi, (\psi \land \chi), (\psi \lor \chi), (\psi \implies \chi)} $ and $ x $ is free in $ \psi $ or $ \chi $, then $ x $ is free in $ \varphi $
%		
%		\item Otherwise $ x $ is not free in $ \varphi $
%		
%	\end{itemize}

	\item If $ \varphi $ has no free variables, $ \varphi $ is a sentence, $ \varphi \in Sent $
	
\end{itemize}

\newpage

\section{Model Theory}

\begin{itemize}
	
	\item A model $ \Model $ requires:
	
	\begin{itemize}
		
		\item A language $ \Lang $
		
		\item A domain $ M $ of objects
		
		\item An interpretation:
		
		\begin{itemize}
			
			\item Constant symbols $ c $ are interpreted by some object from the domain: $ c^\Model \in M $
			
			\item Relation symbols $ R $ (with arity $ n $) are interpreted by a set of tuples of objects within the domain; so $ R^\Model \subseteq \setcomp{\tuple{m_1, \dots, m_n}}{m_1, \dots, m_n \in M} $
			
			\item Function symbols $ f $ with arity $ n $ are interpreted by functions taking some $ m_1, \dots, m_n \in M$, and returning some $ m \in M $
			
		\end{itemize}
		
	\end{itemize}

	\item A model of $ \Lang = \set{a, b, c, \dots, P, Q, R, \dots, f, g, h, \dots} $ would then be:
	\begin{equation*}
	\Model = \tuple{M, a^\Model, b^\Model, c^\Model, \dots, P^\Model, Q^\Model, R^\Model, \dots, f^\Model, g^\Model, h^\Model, \dots}
	\end{equation*}
	
	\item If $ t $ is a closed term of $ \Lang $ and $ \Model $ is a model of $ \Lang $, then the \textit{denotation} $ t^\Model $ is:
	
	\begin{itemize}
		\item If $ t $ is a constant symbol $ c $, $ t^\Model = c^\Model $
		
		\item If $ t = f(t_1, \dots, t_n) $, $ t^\Model = f^\Model (t_1^\Model, \dots, t_n^\Model) $
	\end{itemize}

	\item A sentence $ \varphi $ in some language $ \Lang $ may then be \textit{true} in $ \Model $ ($ \Model \models \varphi $)
	
	
	\item $ \varphi $ is \textit{satisfiable} if there exists some model $ \Model $ in the language $ \Lang $ of $ \varphi $ such that $ \Model \models \varphi $
	
	\item $ \varphi $ is \textit{valid} ($ \models \varphi$) if for every model $ \Model $ in the language $ \Lang $ of $ \varphi $, $ \Model \models \varphi $
	
	\item Given $ \Gamma \subseteq Sent_\Lang $ and $ \varphi \in Sent_\Lang $, we say $ \varphi $ \textit{is a consequence of} $ \Gamma $ ($ \Gamma \models \varphi $) if for every model $ \Model $ of $ \Lang $, if $ \Model \models \gamma $ for all $ \gamma \in \Gamma $, then $ \Model \models \varphi $
	
	\item $ \varphi $ is \textit{derivable} from $ \Gamma $ ($ \Gamma \vdash \varphi $) if:
	
	\begin{itemize}
		
		\item (Ax) $ \varphi $ is an axiom of first order logic
		
		\item (Ass) $ \varphi \in \Gamma $
		
		\item (MP) $ \Gamma \vdash \psi \implies \varphi $ and $ \Gamma \vdash \psi $
		
		\item (UG) If $ \Gamma \vdash \varphi $, then $ \Gamma \vdash \forall y \varphi(y \mapsto x) $ when:
		
		\begin{itemize}
			\item $ x $ is not free in any formula in $ \Gamma $;
			\item $ y = x $; or
			\item $ y $ is not free in $ \varphi $
		\end{itemize}
		
	\end{itemize}

	\item A model can describe a structure \textit{externally} (as in the graph model), or \textit{internally} using sentences true in the intended model
	
	\item A \textit{theory} $ \Gamma $ is a set of sentences closed under consequence (i.e. $ \Gamma \vdash \varphi \implies \varphi \in \Gamma $)
	
	\begin{itemize}
		
		\item It is \textit{consistent} if there is no sentence $ \varphi $ such that $ \Gamma \vdash \varphi \land \lnot \varphi $
		
		\item It is \textit{complete} if for all $ \varphi $ we have either $ \Gamma \vdash \varphi $ or $ \Gamma \vdash \lnot \varphi $
		
		\item It is \textit{categorical} if there is exactly one model $ \Model $ such that $ \Model \models \Gamma $
		
	\end{itemize}
	
	\item A theory is \textbf{algebraic} if it has multiple intended models, and \textbf{non-algebraic} if it has one unique intended model
	
\end{itemize}

\newpage
	\section{The Axioms}

\subsection{Terms \& Definitions}

\begin{itemize}
	
	\item A \textit{term} is well-defined in Set Theory if it \textit{exists} and is \textit{unique}
	
	\item Term: $ x $ is a \textit{subset} of $ y $ ($ x \subseteq y$) if $ \forall z (z \in x \implies z \in y) $
	
\end{itemize}

\subsection{Extensionality}
\begin{equation*}
\forall x \forall y (\forall z (z \in x \iff z \in y) \implies x = y)
\end{equation*}

\begin{itemize}
	
	\item If sets $ x $ and $ y $ have exactly the same members, then they are the same set
	
\end{itemize}

\subsection{Foundation}
\begin{equation*}
\forall x (\exists y \in x \implies \exists z (z \in x \land \forall w (w \in x \implies w \notin z)))
\end{equation*}

\begin{itemize}
	
	\item If $ x $ is non-empty, then it has an $ \in $-minimal member (there is some $ z \in x $ such that every member of $ x $ is not a member of $ z $)
	
\end{itemize}

\subsection{Separation}
\begin{equation*}
\forall x_0 \dots \forall x_n \forall w \exists y \forall z (z \in y \iff z \in w \land \varphi (z, x_0 \dots, x_n))
\end{equation*}

\begin{itemize}
	
	\item Given any set $ w $,  there is a set $ y $ consisting of  exactly the elements $ z $ from $ w $ such that $ \varphi(z) $. The $ \varphi $s are \textit{separated out} from $ w $ to get $ y $
		
	\item Why do we need a $ w $? We would like to do na\"ive comprehension: $ \exists y \forall x (x \in y \iff \varphi(x)) $
	
	\begin{itemize}
		
		\item \textit{(Russell) Any theory including na\"ive comprehension is inconsistent}
		
		%	\begin{proof}
		%		Let $ \varphi(x) = x \notin x $ (i.e. $ \varphi $ says $ x $ is not a member of itself).
		%		
		%		Fix $ r $ such that for all $ x $,  $ x \in r \iff x \notin x $. But then $ r \in r \iff r \notin r $ which is a contradiction.
		%	\end{proof}
		
		\item This also means there is no universal set, i.e. a set $ x $ such that $ \forall y (y \in x \iff y = y) $
		
		%	\begin{proof}
		%		Suppose there is such an $ x $. Then by separation we get $ \RussellSet $.
		%		
		%		But for all $ w $, $ w \in \RussellSet \iff w \notin w $. Since $ x $ contains all of the sets, $ \RussellSet $ is Russell's set which is a contradiction.
		%	\end{proof}
		
	\end{itemize}
	

	
\end{itemize}

\subsection{Pairing}
\begin{equation*}
\forall x \forall y \exists z (x \in z \land y \in z)
\end{equation*}

\begin{itemize}
	
	\item 
	
\end{itemize}

\subsection{Union}
\begin{equation*}
\forall x \exists z \forall y \forall w (y \in w \land w \in x \implies y \in z)
\end{equation*}


\begin{itemize}
	
	\item 
	
\end{itemize}

\subsection{Powerset}
\begin{equation*}
\forall x \exists y \forall z (z \subseteq x \implies z \in y)
\end{equation*}

\begin{itemize}
	
	\item 
	
\end{itemize}


\subsection{Replacement}
\begin{equation*}
aa
\end{equation*}

\begin{itemize}
	
	\item 
	
\end{itemize}

\subsection{Infinity}
\begin{equation*}
\exists x (\exists y y \in x \land \forall z (z \in x \implies \set{z} \in x))
\end{equation*}

\begin{itemize}
	
	\item 
	
\end{itemize}

\subsection{Choice}
\begin{equation*}
aa
\end{equation*}

\begin{itemize}
	
	\item 
	
\end{itemize}












\section{Models, Structures \& Sequences}

\section{The Ordinals}

\section{Transfinite Induction and Recursion}

\section{The Cardinals}

\section{Infinite Cardinals \& The Axiom of Choice}
\end{document}