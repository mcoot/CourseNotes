\section{Partial Recursive Functions}

\subsection{Partial Functions, Definition by Composition \& Primitive Recursion}

\begin{itemize}
	
	\item Classes of functions:
	
	\begin{itemize}
		
		\item Let $ P $ be the set of partial functions, $ P = \setcomp{f}{f \text{ is a partial function } \Nat^n \to \Nat \text{ for some } n > 0} $
		
		\item Let $ T $ be the set of total functions, $ T = \setcomp{f \in P}{f \text{ is total}} $
		
		\item A \textit{class} of functions means a subset of $ P $, and a class of total functions means a subset of $ T $
		
		\item Goal: build a class of functions which we might call `computable'
		
	\end{itemize}
	
	\item Let $ g: \Nat^r \to \Nat, h_1 \dots h_r: \Nat^n \to \Nat $ be partial functions.
	
	Then the partial function $ f: \Nat^n \to \Nat $ obtained from $ g, h_1, \dots, h_r $ by composition is defined by:
	\begin{equation*}
	f(\vec{x}) = g(h_1(\vec{x}), \dots, h_r(\vec{x}))
	\end{equation*}
	\begin{itemize}
		\item We write $f = g \circ (h_1, \dots, h_r) $
	\end{itemize}
	
	\item Let $ g: \Nat^n \to \Nat, h: \Nat^{n+2} \to \Nat $ be partial functions.
	
	Then the partial function $ f: \Nat^{n+1} \to \Nat $ obtained from $ g $ and $ h $ by primitive recursion is defined by:
	\begin{align*}
	&f(\vec{x}, 0) = g(\vec{x})\\
	&f(\vec{x}, y + 1) = h(\vec{x}, y, f(\vec{x}, y))
	\end{align*}
	\begin{itemize}
		\item For a given $ \vec{x} $, $ f(\vec{x}, y) $ is defined for no $ y $, for all $ y $, or for $ 0 \le y \le r $ for some $ r \in \Nat $
		\item Where the `counter' parameter is placed does not matter - it could equally be at the start
	\end{itemize}
	 
\end{itemize}

\subsection{Primitive Recursive Functions}

\begin{itemize}
	
	\item We define the \textit{initial functions} to be the following functions:
	
	\begin{itemize}
		
		\item The zero function $ z: \Nat \to \Nat $, such that $ z(x) = 0 $ for all $ x \in \Nat $
		
		\item The successor function $ \sigma: \Nat \to \Nat $, such that $ \sigma(x) = x + 1 $ for all $ x \in \Nat $
		
		\item The projection functions $ \pi_{i, n}: \Nat^n \to \Nat $, where for $ n \ge 1 $ and $ 1 \le i \le n $, $ \pi_{i, n}(x_1, \dots, x_n) = x_i $
		
	\end{itemize}

	\item A class $ \Class $ of total functions is \textit{primitively recursively closed} if:
	
	\begin{itemize}
		
		\item $ \Class $ contains all the initial functions
		
		\item $ \Class $ is closed under composition
		
		\item $ \Class $ is closed under primitive recursion
		
	\end{itemize}

	\item The smallest primitively recursively closed class (i.e. the intersection of all prim. rec. closed classes) is called \textit{the class of primitive recursive functions}
	
	\item Example: addition function $ S: \Nat^2 \to \Nat $, such that $ S(x, y) = x + y $
	\begin{align*}
	S(x, 0) &= g(x), g = \pi_{1, 1}\\
	S(x, y  + 1) &= S(x, y) + 1\\
				 &= \sigma(S(x, y))\\
				 &= h(x, y, S(x, y)), h = \sigma \circ \pi_{3, 3}
	\end{align*}
	
	\newpage
	
	\item Useful tips for showing a function is in a primitively recursively closed class $ \Class $:
	
	
	\begin{itemize}
		
		\item Given $ f: \Nat^n \to \Nat $ is in $ \Class $
		
		If $ g: \Nat^m \to \Nat $ is defined by $ g(x_1, \dots, x_m) = f(y_1, \dots, y_n) $ where each $ y_i $ is either a constant or $ x_j $ for some $ j $, then $ g \in \Class $ -- lets you manipulate arity
		
		\item To show a unary function $ f: \Nat \to \Nat $ is in $ \Class $ by primitive recursion, define $ f': \Nat^2 \to \Nat $ such that $ f'(x, y) = f(y) $; then, if $ f' $ can be shown to be in $ \Class $, $ f $ will be also
		
		\item Let $ a \in \Nat $ and $ h: \Nat^2 \to \Nat $ be in $ \Class $
		
		Then, for $ f: \Nat \to \Nat $, if $ f(0) = a $ and $ f(y + 1) = h(y, f(y)) $, $ f \in \Class $
		
	\end{itemize}

	\item A \textit{primitive recursive definition} of $ f: \Nat^n \to \Nat $ is a finite sequence of functions $ f_0, f_1, \dots, f_k = f $, where for each $ i $:
	
	\begin{itemize}
		
		\item $ f_i $ is initial; or
		
		\item $ f_i $ is obtained from composition of some functions $ f_j $, $ j < i $; or
		
		\item $ f_i $ is obtained by primitive recursion from two of $ f_j $, $ j < i $
		
	\end{itemize}

	\item Example: addition function $ S $ can be defined by $ \pi_{1, 1}, \pi_{3, 3}, \sigma, \sigma \circ \pi_{3, 3} $
	
	\item The class $ \Class_1 $ of primitive recursive functions is the same as the class $ \Class_2 $ of functions that have a primitive recursive definition (seems trivial, but isn't!)
	
	Prove by showing $ \Class_1 \subseteq \Class_2 $ (i.e. $ \Class_2 $ is prim. rec. closed) and that $ \Class_2 \subseteq \Class_1 $ (i.e. $ \Class_2 $ is contained in any prim. rec. closed class)
	
	\item Let $ \Class $ be a prim. rec. closed class, and let $ g: \Nat^{n+1} \to \Nat $ be in $ \Class $; then the functions $ f_1: \Nat^{n+1} \to \Nat $ and $ f_2: \Nat^{n+1} \to \Nat $ defined by:
	
	$ f_1(\vec{x}, y) = \sum_{t=0}^{y} g(\vec{x}, t) $
	
	$ f_2(\vec{x}, y) = \prod_{t=0}^{y} g(\vec{x}, t) $
	
	are also in $ \Class $
	
	\item Useful prim. rec. functions:
	
	\begin{itemize}
		\item Proper subtraction $ x \monus y = max\set{x - y, 0} $
		
		\item Sign $ sg(x) = \begin{cases}
		0 \text{ if } x = 0\\
		1 \text{ if } x \ge 0
		\end{cases} $
		
	\end{itemize}
	
\end{itemize}

\newpage

\subsection{Predicates}

\begin{itemize}
	\item A predicate $ P(x_1, \dots, x_n) $ of $ n $ variables is a statement concerning $ x_i \in \Nat$ which is either true or false
	
	\item We can identify $ P $ with the set $ A_P = \setcomp{\vec{x} \in \Nat^n}{P(\vec{x}) \text{ is true}} $
	
	\subitem E.g. $ P(x, y) $ means ``$ x $ divides $ y $", so $ A_P = \set{(1, 6), (2, 6), (3, 6), (6, 6), (1, 3) \dots} $
		
	\item The \textit{characteristic function} of a set $ \chi_A: \Nat^n \to \set{0, 1} $ of $ A \subseteq \Nat^n $ is defined by:
	\begin{equation*}
	\chi_A(\vec{x}) = \begin{cases}
	1 \text{ if } \vec{x} \in A\\
	0 \text{ if } \vec{x} \notin A
	\end{cases}
	\end{equation*}
	\item For a predicate $ P $, we define $ \chi_P $ to be $ \chi_{A_P} $
	
	\item Let $ \Class $ be a prim. rec. closed class; then a subset $ A \subseteq \Nat^n $ is in $ \Class $ if $ \chi_A \in \Class $
	
	\subitem So a predicate $ P $ of $ n $ variables is in $ \Class $ if $ \chi_P \in \Class $
	
	\item If $ A, B \subseteq \Nat^n $ are in $ \Class $, then $ A \cup B $, $ A \cap B $ and $ \Nat^n \setminus A $ are in $ C $
	
	\subitem So if $ P, Q $ are predicates of $ n $ variables in $ \Class $, $ P \lor Q $, $ P \land Q $ and $ \lnot P $ are in $ \Class $
	
	\subitem Proof: $ \chi_{A \cup B}(x) = sg(\chi_A(x) + \chi_B(x)) $, ~~~ $ \chi_{A \cap B} = \chi_A(x) \cdot \chi_B(x) $, ~~~ $ \chi_{\Nat^n \setminus A}(x) = 1 \monus \chi_A(x) $
	
	\item The predicates $ x = y $, $ x \ne y $, $ x \le y $, $ x < y $, $ x \ge y $, $ x > y $ are prim. rec.
	
	\subitem Proof: Note that $ \chi_{\ne} (x, y) = sg(\abs{x - y}) $ and $ \chi_{>} (x, y) = sg(x \monus y) $
	
	\item Bounded quantifiers:
	
	Assume $ P $ is a pred. of $ n + 1 $ variables in $ \Class $; then $ Q, R $ of $ n + 1 $ variables defined below are in $ \Class $:
	
	$ Q(x_1, \dots x_n, z) $ is true if and only if $ \exists_{y \le z} (P(x_1, \dots, x_n, y) \text{ is true})$
	
	$ R(x_1, \dots x_n, z) $ is true if and only if $ \forall_{y \le z} (P(x_1, \dots, x_n, y) \text{ is true})$
	
	\subitem Proof: $ \chi_Q (\vec{x}, z) = sg(\sum_{y=0}^{z} \chi_P(\vec{x}, y)) $, and $ \chi_R (\vec{x}, z) = \prod_{y=0}^{z} \chi_P(\vec{x}, y) $
	
\end{itemize}

\subsection{More Primitive Recursive Functions}

\subsubsection{Bounded Minimisation}

Let $ P $ be a pred. of $ n + 1 $ variables. Define $ f: \Nat^{n+1} \to \Nat $ by:
\begin{equation*}
f(\vec{x}, z) = \begin{cases}
\text{the least } y \le z \text{ s.t. } P(\vec{x}, y) \text{ is true}\\
z + 1 \text{ if no such } y \text{ exists}
\end{cases}
\end{equation*}
Then, $ f(\vec{x}, z) = \mu ~ {y \le z} ~ P(\vec{x}, y) $, called \textit{bounded minimisation}. We have that if $ P \in \Class$ (a prim. rec. closed class), then $ f $ is in $ \Class $.

\begin{proof}
	
Define $ g(\vec{x}, t) = \prod_{y=0}^{t} sg(1 \monus \chi_P(\vec{x}, y) \text{ is true})$. Note that $ g(\vec{x}, t) =  \begin{cases}
0 \text{ if } \exists_{y \le t} P(x, y) \text{ is true}\\
1 \text{ if } \forall_{y \le t} P(x, y) \text{ is false}
\end{cases} $

Let $ y \le z $ be the least s.t. $ P(\vec{x}, y) $ is true.

Then the values of $ g $ look like: \begin{tabular}{r | c c c c c c c c}
	$ t $ & 0 & 1 & $ \dots $ & $ y - 1 $ & $ y $ & $ y + 1 $ & $ \dots $ & $ z $ \\ \hline
	$ g(\vec{x}, t) $ & 1 & 1 & $ \dots $ & 1 & 0 & 0 & $ \dots $ & 0 
\end{tabular}

Let $ f(\vec{x}, z) = \sum_{t=0}^{z} g(\vec{x}, t) $, then we will have $ f $ as required for bounded minimisation. If there is no such $ y $, then by the definition of $ g $ we would have $ f(\vec{x}, z) = z + 1 $

\end{proof}

\subsubsection{Definition By Cases}

Let $ f_1, \dots, f_k : \Nat^n \to \Nat $ be in  prim. rec. closed $ \Class $ and let $ P_1, \dots, P_k $ be predicates in $ \Class $ of $ n $ variables. Suppose that for each $ \vec{x} \in \Nat^n $ \textit{exactly} one of $ P_1(\vec{x}), \dots, P_k(\vec{x}) $ is true. Define $ f: \Nat^n \to \Nat $ by:
\begin{equation*}
f(\vec{x}) = f_i(\vec{x}) \text{ if } P_i(\vec{x}) \text{ is true}
\end{equation*}
Then $ f $ is in $ \Class $.

\begin{proof}

$ f(\vec{x}) = f_1(\vec{x}) \cdot \chi_{P_1}(\vec{x}) + \dots + f_k(\vec{x}) \cdot \chi_{P_k}(\vec{x}) $

\end{proof}

\subsubsection{Iteration}

Let $ X $ be a set, with a partial function $ f: X \to X $. The \textit{iterate} of $ f $ is the partial function $ F: X \times \Nat \to X $ defined by:
\begin{align*}
&F(x, 0) = x\\
&F(x, n + 1) = f(F(x, n))
\end{align*}
We have a notion of a function $ f: \Nat^n \to \Nat $ being in a class $ \Class $. This can be extended to functions $ f: \Nat^n \to \Nat^k $ by saying that $ f $ is in $ \Class $ if $ \pi_{i, k} \circ f $ is in $ \Class $ for each $ 1 \le i \le k $.

A class $ \Class $ is closed under iteration if, whenever $ f: \Nat^n \to \Nat^n $ is in $ \Class $, then its iterate $ F: \Nat^{n+1} \to \Nat^n $ is in $ \Class $.

Let $ \Class $ be a prim. rec. closed class. Then if $ f: \Nat^n \to \Nat^n $ is in $ \Class $, its iterate $ F: \Nat^{n+1} \to \Nat^n $ is also in $ \Class $. So any prim. rec. closed class is closed under iteration.

\begin{proof}

This shows only the $ n = 1 $ case.

Define $ f': \Nat^3 \to \Nat $ by $ f'(x, y, z) = f(z) $, which is in $ \Class $.

Then the iterate of $ f $ is defined by:

\begin{align*}
&F(z, 0) = z\\
&F(z, y + 1) = f(F(z, y)) = f'(x, y, F(x, y))
\end{align*}

This is defined by primitive recursion, so $ F $ is in $ \Class $.

\end{proof}

\newpage

\subsection{Recursive and Partial Recursive Functions}

\subsubsection{Minimisation}

Let $ f: \Nat^{n+1} \to \Nat $ be a partial function. The function obtained from $ f $ by \textit{minimisation} is the partial function $ g: \Nat^n \to \Nat $ defined by
\begin{equation*}
g(\vec{x}) = \begin{cases}
r &\text{ if } f(\vec{x}, r) = 0 \text{ and for } s < r, f(\vec{x}, s) \text{ is defined and not 0}\\
\textit{undefined} &\text{ otherwise}
\end{cases}
\end{equation*}
We write $ g(\vec{x}) = \mu y (f(\vec{x}, y) = 0) $. It is also called the $ \mu $-operator or unbounded search operator. The function $ g $ may be partial, even if $ f $ is total, and vice versa.

Note that it is not \textit{quite} accurate to say $ g(x) = \mu y (f(\vec{x}, y) = 0) $ is the least $ y $ s.t. $ f(x, y) = 0 $; if there is some least $ y $ s.t. $ f(x, y) = 0 $, but $ f(x, s) $ is undefined for some $ s < y $, then $ g(x) $ is undefined.

\subsubsection{The Class of Recursive Functions}

\begin{itemize}
	
	\item A total function $ f(\vec{x}, y) $ is \textit{regular} if for any $ \vec{x} \in \Nat^n $, there exists $ y \in \Nat $ such that $ f(\vec{x}, y) = 0 $
	
	\item The regular functions are exactly those to which we can apply minimisation and end up with a total function
	
	\item The function $ g $ is obtained from $ f $ by \textit{regular minimisation} if $ g(\vec{x}) = \mu y (f(\vec{x}, y) = 0) $ where $ f $ is regular
	
	\item The \textit{class of recursive functions} is the smallest class $ \Class $ of total functions which is primitively recursively closed and is closed under regular minimisation
	
	\item Note that there are recursive functions which are \textit{not} primitive recursive
	
	\item Example: the two-argument Ackermann function defined by:
	\begin{equation*}
	A(m, n) = \begin{cases}
	n + 1 &\text{ if } m = 0\\
	A(m - 1, 1) &\text{ if } m > 0 \text{ and } n = 0\\
	A(m - 1, A(m, n - 1)) &\text{ if } m > 0 \text{ and } n > 0
	\end{cases}
	\end{equation*}
	
	\item The \textit{class of partial recursive functions} is the smallest class of partial functions which contains the initial functions, and is closed under composition, primitive recursion and minimisation
	
	\begin{itemize}
		\item Note that this is \textit{not} a primitively recursively closed class -- that term only applies to a class of total functions
	\end{itemize}
	
\end{itemize}