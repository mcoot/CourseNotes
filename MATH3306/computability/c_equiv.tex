\section{Equivalence of Partial Recursive and TM Computable \mbox{Functions}}

A key theorem is that all partial recursive functions are Turing Machine computable, and vice versa.

\subsection{TM Computable Functions Are Partial Recursive}

Recall that a partial function $ f: \Nat^n \to \Nat $ is TM computable if $ f = \varphi_{T, n} $ for some numerical TM $ T $.

Let $ T = (Q, F, A, I, \tau, q_0) $ be a numerical Turing machine (i.e. deterministic, $ F = I = \emptyset $, $ A = \set{0, 1} $).

Recall that $ \varphi_{T, n}(\vec{x}) = \begin{cases}
y &\text{if the computation starting with } (q_0, \underline{0}1^{x_1} \dots 01^{x_n}) \text{ halts with } (q, \underline{0}1^y)\\
\textit{undefined} &\text{otherwise}
\end{cases} $

It is convenient to modify $ T $ slightly. Add two new states $ p $ and $ h $, and the transitions:

\begin{itemize}
	
	\item $ (q, a, p, a, L) $ for all $ (q, a) \in Q \times A $ s.t. no element in $ \tau $ starts with $ (q, a) $
	
	\item $ (p, a, h, a, R) $ for all $ a \in A $ (i.e. for $ a = 0 $ and $ a = 1 $)
	
	\item $ (h, a, p, a, L) $ for all $ a \in A $
	
\end{itemize}

Call the new machine $ T' $, so $ Q' = Q \cup \set{p, h} $, with $ C' $ being the set of configurations.

Then $ T' $ is still deterministic, and transitions have the form:
\begin{equation*}
(q, a, N(q, a), R(q, a), D(q, a)) \in Q' \times A \times Q' \times A \times \set{L, R}
\end{equation*}
where $ N, R, D $ are functions on $ Q' \times A $.

Then, we number the states such that $ Q = \set{0, 1, \dots, r - 1} $, where $ h = 0 $ and $ p = 1 $. We encode $ L = 0 $ and $ R = 1 $.

Now, $ Q' \times A $ is a finite subset of $ \Nat^2 $; put $ N(x, y) = R(x, y) = D(x, y) = 0 $ for $ (x, y) \in \Nat^2 \setminus (Q' \times A) $. Then, $ N, R, D $ are primitive recursive functions $ \Nat^2 \to \Nat $.

Define $ Code: C' \to \Nat $ by $ Code(q, a, \alpha, \beta) = 2^q 3^a 5^{\sigma(\alpha)} 7^{\sigma(\beta)} $, where $ \sigma $ encodes a function in the binary representation of an integer:
\begin{equation*}
\sigma(f) = f(0) + 2 \cdot f(1) + 2^2 \cdot f(2) + \dots
\end{equation*}
Then $ Code $ is an injective (one-to-one) function.

There is a primitive recursive function $ Next: \Nat \to \Nat $ s.t. $ Next(Code(c)) = Code(\delta(c)) $, for $ c \in C' $ where $ \delta $ is the transition function of $ T' $.

\begin{proof}

Let $ c = (q, a, \alpha, \beta) $; let $ x \in \Nat = Code(c) = 2^q 3^a 5^{\sigma(\alpha)} 7^{\sigma(\beta)} $.

Then, we express $ Next(x) = Code(\delta(c)) $ in terms of $ x $.

First, note that $ q = \log_2 x $ and $ a = \log_3 x $ (here $ \log $ simply retrieves the exponents, it is not the normal logarithm function from calculus/analysis).

We have that $ N(q, a) = N(\log_2 x, \log_3 x) $; the $ \log $ and $ N $ functions are primitive recursive.

There are then two cases, moving left or right:

\newpage

\subsubsection{Move left - $ D(q, a) = 0 $}

We have that $ \delta(c) = (q', a', \alpha', \beta') $, where $ q' = N(q, a) $ and $ a' = \beta(0) $.

$ Next(x) = Code(\delta(c)) = 2^{N(q, a)} 3^{\beta(0)} 5^{\sigma(\alpha')} 7^{\sigma(\beta')} $

$ \beta(0) = rem(2, \log_7 (x)) $ where $ rem $ is the remainder function (which is prim. rec.).

$ \sigma(\alpha') = R(q, a) + 2 \alpha(0) + 2^2 \alpha(1) + \dots = R(\log_2 x, \log_3 x) + 2 \log_5 x$, where $ R $ is prim. rec.

$ \sigma(\beta') = \beta(1) + 2 \beta(2) + 2^2 \beta(3) + \dots = quo(2, \sigma(\beta)) = quo(2, \log_7 x) $, where $ quo $ is the quotient / `integer division' function (which is prim. rec.).

\subsubsection{Move right - $ D(q, a) = 1 $}

In this case we have that $ \delta(c) = (q', a', \alpha', \beta') $, where $ q' = N(q, a) $ and $ a' = \alpha(0) $.

$ \alpha(0) = rem(2, \log_5 x) $

$ \sigma(\alpha') = quo(2, \log_5 x) $

$ \sigma(\beta') = R(\log_2 x, \log_3 x) + 2 \log_7 x $

\subsubsection{Conclusion}

We can combine both cases using $ E(x) = D(\log_2 x, \log_3 x) $. This gives us the functions:

$ F_1(x) = N(\log_2 x, \log_3 x) $

$ F_2(x) = (1 \monus E(x)) \cdot rem(2, \log_7 x) + E(x) rem(2, \log_5 x) $

$ F_3(x) = (1 \monus E(x)) \cdot (R(\log_2 x \log_3 x) + 2 \log_5 x) + E(x) \cdot quo(2, \log_5 x) $

$ F_4(x) = (1 \monus E(x)) \cdot quo(2, \log_7 x) + E(x) \cdot(R(\log_2 x, \log_3 x) + 2 \log_7 x) $

Clearly each of these is a composition of primitive recursive functions, and so each is primitive recursive.

Then, $ Next(x) = 2^{F_1(x)} 3^{F_2(x)} 5^{F_3(x)} 7^{F_4(x)} $. This is a composition of exponentiation and functions known to be primitive recursive, so $ Next(x) $ is also primitive recursive.

\end{proof}

Recall that if $ f: \Nat \to \Nat $ is primitive recursive, then its iterate $ F: \Nat^2 \to \Nat $ is also prim. rec.

Let $ \bar{\delta} $ be the iterate of $ \delta $. If $ Comp $ is the iterate of $ Next $, then $ Comp(Code(c), t) = Code(\bar{\delta}(c, t)) $ for any $ c \in C' $ and $ t \in \Nat $.
 
\begin{proof}
	Use induction on $ t $.
	
	First, $ Comp(Code(c), 0) = Code(c) = Code(\bar{\delta}(c, 0)) $.
	
	Now, assume that $ Comp(Code(c), t) = Code(\bar{\delta}(c, t)) $ holds.
	
	Then, we have:
	\begin{align*}
	Comp(Code(c), t + 1) &= Next(Comp(Code(c), t))\\
						 &= Next(Code(\bar{\delta}(c, t)))\\
						 &= Code(\delta(\bar{\delta}(c, t)))\\
						 &= Code(\bar{\delta}(c, t + 1))
	\end{align*}
\end{proof}

\newpage

Define the function $ In_{T, n}: \Nat^n \to C' $, such that $ In_{T, n}(\vec{x}) $ returns the initial configuration of $ T $ when started with the tape described by $ Tape(\vec{x}) $.

\textbf{Main theorem}: the function $ \varphi_{T, n} $ is partial recursive.

\begin{proof}
	Note $ \varphi_{T, n}(\vec{x}) = \begin{cases}
	y &\text{ if } \exists t \in \Nat \text{ s.t. } \bar{\delta}(In_{T, n}(\vec{x}), t) = (h, \underline{0} 1^y) \text{ for some } y \in \Nat\\
	\textit{undefined} &\text{otherwise}
	\end{cases} $
	
	Also note that $ Code(h, \underline{0} 1^y) = 2^0 3^0 5^{1 + 2 + 2^2 + \dots + 2^{y - 1}} 7^0 = 5^{2^y - 1}$.
	
	If $ \bar{\delta}(In_{T, n}(\vec{x}), t) = (h, \underline{0} 1^y) $ for some $ t, y \in \Nat $, then we have that:
	
	$ Comp(Code(In_{T, n}(\vec{x})), t) = Code(\bar{\delta}(In_{T, n}(\vec{x}), t)) = 5^{2^y - 1} $
	
	Define $ \psi: \Nat^{n+1} \to \Nat $ by $ \psi(\vec{x}, t) = Comp(Code(In_{T, n}(\vec{x})), t) $.
	
	The composition $ Code(In_{T, n}(\vec{x})) $ is primitive recursive (from assignments), and $ Comp $ is primitive recursive since it is the iterate of the primitive recursive $ Next $.
	
	Then, $ \varphi_{T, n}(\vec{x}) = \begin{cases}
	\log_2 (1 + \log_5(\psi(\vec{x}, t))) &\text{ for any } t \in \Nat \text{ s.t. } \psi(\vec{x}, t) = 5^{2^y - 1} \text{ for some } y\\
	\textit{undefined} &\text{otherwise}
	\end{cases} $

\end{proof}