\section{Turing Machines}

\subsection{Definition}

\begin{itemize}
	
	\item A Turing machine is a tuple $ T = (Q, F, A, I, \tau, q_0) $
	
	\begin{itemize}
		
		\item $ Q $ is a finite set of states
		
		\item $ F \subseteq Q $ is the set of final states
		
		\item $ A $ is a finite set, the tape alphabet, with a distinguished blank symbol $ B \in A $
		
		\item $ I $ is a subset of $ A \setminus \set{B} $, the input alphabet
		
		\item $ \tau \subseteq Q \times A \times Q \times A \times \set{L, R} $ is the set of transitions
		
		\item $ q_0 \in Q $ is the initial state
				
	\end{itemize}
	
	\item As in an FSA, non-determinism is allowed
	
	\item The tape is infinite in both directions, but only ever contains a finite number of non-blank symbols
	
	\item A \textit{tape description} for $ T $ is a triple $ (a, \alpha, \beta) $ with $ a \in A $, and $ \alpha: \Nat \to A $ and $ \beta: \Nat \to A $ being functions with $ a(n) = B $ and $ B(n) = B $ for all but finitely many $ n \in \Nat $
	
	\begin{itemize}
		\item So the tape looks like: $ \dots BBB \beta(l) \beta(l - 1) \dots \beta(0) \underline{a} \alpha(0) \alpha(1) \dots \alpha(r) BBB \dots $, with $ l, r \in \Nat $
	\end{itemize}
	
	\item A \textit{configuration} of $ T $ is a tuple $ (q, a, \alpha, \beta) $ where $ q \in Q $ and $ (a, \alpha, \beta) $ is a tape description
	
	\item If $ c = (q, a, \alpha, \beta) $ is a configuration, a configuration $ c' $ is obtained (reachable) from $ c $ by a single move if one of the following holds:
	
	\begin{itemize}
		\item $ (q, a, q', a', L) \in \tau $ and $ c' = (q', \beta(0), \alpha', \beta') $ where:
		$ \alpha'(0) = a', \alpha'(n) = \alpha(n - 1), n > 0 $ and $ \beta'(n) = \beta(n + 1), n \ge 0 $, or
		
		\item $ (q, a, q', a', R) \in \tau $ and $ c' = (q', \alpha(0), \alpha', \beta') $ where:
		$ \alpha'(n) = \alpha(n + 1), n \ge 0 $ and $\beta'(0) = a', \beta'(n) = \beta(n - 1), n > 0 $
	\end{itemize}

	\item A \textit{computation} of $ T $ is a finite sequence of configurations $ c_1, \dots, c_n = c' $ where $ n \ge 1 $ and $ c_{i+1} $ is obtained from $ c_i $ by a single move, for $ 1 \le i \le n - 1 $
	
	\item A configuration is \textit{terminal} if no configuration is reachable from it
	
	\item A computation halts if $ c' $ is terminal (i.e. there is no configuration reachable from $ c' $)
	
	\item We may write $ c \turingcomputes{T} c' $ if there is a computation starting at $ c $ and ending at $ c' $
	
\end{itemize}

\subsection{Turing Machine as Language Recogniser}

\begin{itemize}
	
	\item For $ w = a_1 \dots a_n \in A^* $, let $ c_w = (q_0, \underline{a_1} \dots a_n) $ (recall $ \underline{a_1} \dots a_n $ is a tape description $ (a, \alpha, \beta) $)
	
	\item If $ w = \varepsilon $, we put $ c_w = (q_0, \underline{B}) $
	
	\item The TM $ T $ \textit{accepts} if $ c_w \turingcomputes{T} c' $ for some $ c' = (q, a, \alpha, \beta) $ with $ q \in F $
	
	\item The language recognised by $ T $ is $ \Lang(T) = \setcomp{w \in I^*}{w \text{ is accepted by } T } $
	
	\item Note that $ \Lang(T) $ is a language over $ I $ rather than over $ A $
	
	\item $ T $ is deterministic if for every $ (q, a) \in Q \times A $ there is \textit{at most one} element of $ \tau $ starting with $ (q, a) $
	
	\item Then, there is at most one config $ c' $ obtained from $ c $ by a single move; set $ \delta(c) = c' $
	
	\item $ \delta: C \to C $ is then a partial function
	
\end{itemize}

\clearpage

\subsection{Numerical Turing Machines: TMs as Function Calculators}

\begin{itemize}
	
	\item We want to use TMs to describe a partial function $ f: \Nat^n \to \Nat $
	
	\item A \textit{numerical TM} is a deterministic TM $ T = (Q, F, A, I, \tau, q_0) $ with:
	
	\begin{itemize}
		
		\item $ F = I = \emptyset $
		
		\item $ A = \set{0, 1} $, with $ 0 $ as the blank symbol
		
	\end{itemize}

	\item In a numerical TM, the final states $ F $ and input alphabets $ I $ are not relevant
	
	\item For $ \vec{x} = (x_1, \dots, x_n) \in \Nat^n $, define the tape description $ Tape(\vec{x}) = \underline{0} 1^{x_1} 0 1^{x_2} 0 \dots 0 1^{x_n} $
	
	\item Define the partial function $ \varphi_{T, n}: \Nat^n \to \Nat $ as follows:
	
	\begin{itemize}
		
		\item Let $ \vec{x} \in \Nat^n $ be given
		
		\item The initial config of $ T $ is $ (q_0, Tape(\vec{x})) $
		
		\item If $ T $ halts with tape $ \underline{0} 1^{y} = Tape(y) $ for some $ y \in \Nat $, then $ \varphi_{T, n}(\vec{x}) = y $
		
		\item Otherwise, $ \varphi_{T, n} $ is undefined
		
	\end{itemize}

	\item If $ f: \Nat^n \to \Nat = \varphi_{T, n} $ for some numerical TM $ T $, then $ f $ is \textit{TM computable}  
	
	\item Note that when considering TMs as language recognisers, halting is regarded as an error -- but for a numerical TM, it is fine \textit{so long as} it ends with a configuration of the form $ (q, \underline{0}1^y) $ with $ y \in \Nat $
	
	\item Example: an addition function $ S: \Nat^2 \to \Nat $
	
	\begin{tikzpicture}
	\node[circle,thick,draw] (q0) at (0, 0) {$ - $};
	\node[circle,thick,draw] (q1) at (2, 0) {$ q_1 $};
	\node[circle,thick,draw] (q2) at (4, 0) {$ q_2 $};
	\node[circle,thick,draw] (q3) at (6, 0) {$ q_3 $};
	
	\draw[edge] (q0) to node[above] {0/R/0} (q1);
	\draw[edge] (q1) to node[above] {1/R/0} (q2);
	\draw[edge] (q2) to node[above] {0/L/1} (q3);
	
	\draw[edge] (q2) to[loop above] node[above] {1/R/1} (q2);
	\draw[edge] (q3) to[loop above] node[above] {1/L/1} (q3);
	\end{tikzpicture}
	
	\item Ultimate theorem: All TM computable functions are partial recursive, and conversely all partial recursive functions are TM computable
	
\end{itemize}
