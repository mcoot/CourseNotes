\section{Procedures}

\begin{itemize}
	
	\item A procedure is a named block of code used for structure and to enable reuse
	
	\item Given $ \Procedure ~ R() \defeq S $ and $ w : [P, Q] \refsto S $, we have that $ w : [P, Q] \refsto R $
	
	\item The \textit{formal} parameter is used in the function and the \textit{actual} parameter is what's passed in
	
	\item Parameter types
	
	\begin{itemize}
		
		\item $ \Value $
		
		\begin{itemize}
			
			\item Sets the formal parameter to the value of a variable or expression when the procedure runs
			
			\item Modifying the formal parameter in the procedure doesn't affect the actual parameter
			
			\item Given $ \Procedure ~ R(\Value ~ z) \defeq S $  and $ w, z : [P, Q] \refsto S$:\\
			$ ~~~~ w : [P[z \backslash a], Q[z_0 \backslash a_0]] \refsto R(a) $ where $ a_0 = a[w \backslash w_0] $
			
			\item The postcondition $ Q $ should not contain $ z $ since it is local to $ R $
				
		\end{itemize}
		
		\item $ \Result $
		
		\begin{itemize}
			
			\item The actual parameter takes the value of the formal parameter when the procedure terminates
			
			\item The actual parameter must be a variable, not an expression and its initial value is not defined
			
			\item Given $ \Procedure ~ R(\Result z) \defeq S $ and $ w, z : \refsto S $:\\
			$ ~~~~ w : [P, Q[z \backslash a]] \refsto R(a) $
			
			\item The precondition $ P $ should not contain $ z $, and the postcondition $ Q $ should not contain $ z_0 $
			
		\end{itemize}
		
		\item $ \ValueResult $
		
		\begin{itemize}
			
			\item The formal parameter takes the value of the actual parameter when the procedure starts
			
			\item The actual parameter takes the value of the formal parameter when the procedure terminates
			
			\item Given $ \Procedure ~ R(\ValueResult ~ z) \defeq S $ and $ w, z : [P, Q] \refsto S $:\\
			$ ~~~~ w, a : [P[z \backslash a], Q[z_0, z \backslash a_0, a]] \refsto R(a) $
			
			\item There are no constraints on how $ z $ and $ z_0 $ may appear in $ P $ and $ Q $
			
		\end{itemize}
	
		\item A procedure may have multiple parameters of different types
		
		\begin{itemize}
			
			\item e.g. $ \Procedure ~ R(\Result ~ x, y; \Value ~ z) \defeq x, y := 0, z + 1 $
			
		\end{itemize}
		
	\end{itemize}

	\item Introducing procedures and procedure calls when refining
	
	\begin{enumerate}
		
		\item Identify a suitable specification $ x, y, z : [P, Q] $ and choose a name $ R $
		
		\item Identify parameters and their types
		
		\begin{itemize}
			
			\item Variables in $ P $ only are likely $ \Value $ parameters
			
			\item Variables in $ Q $ only are likely $ \Result $ parameters
			
			\item Variables in both are likely $ \ValueResult $ parameters
			
		\end{itemize}
		
		$ \Procedure ~ R(\Value ~ x; \Result ~ y; \ValueResult ~ z) \defeq x, y, z : [P, Q] $
		
		\item If the formal parameter appears only in the precondition, use a $ \Value $ parameter
		
		\item Refine the body of $ R $ to code
		
		\item Refine the main program with variables $ a, b, c $ to the specification:\\
		$ ~~~~ b, c : [P[x, z \backslash a, c], Q[x_0, y, z_0, z \backslash a_0, b, c_0, c]]$
		
		\item Replace the above specification with $ R(a, b, c) $
		
	\end{enumerate}
	
\end{itemize}

\newpage

\subsection{Recursion}

\begin{itemize}
	
	\item Procedures may be called recursively (i.e. from within themselves), as per any procedure call
	
	\item Need to use a variant $ V $ to ensure the recursion reaches a base case
	
	\begin{itemize}
		
		\item The variant may refer to variables including parameters (aside from $ \Result $ parameters)
		
		\item Given procedure $ R $ with specification $ w : [P, Q] $, let $ V = N $ when the procedure is first called
		
		\item Then to refine to a call outside the function, $ R $ has specification $ w : [P \land (V = N), Q] $
		
		\item To refine to a recursive call within $ R $, we require the specification $ w : [P \land (0 \le V < N), Q] $
		
	\end{itemize}
	
\end{itemize}

$ ~ $

Example: $ \Procedure ~ Factorial(\Value ~ n, \Result ~ f) \defeq n : f [n \ge 0, f = n_0!] $.

Let $ n $ be the variant and introduce $ n = N $ into the precondition (for the initial call).\\
\form{n, f : [n \ge 0 \land n = N, f = n_0!]}
\hint{\refsto}{Selection: $ n \ge 0 \land n = N \entails n = 0 \lor n > 0 $}
\gcl{0}{\If ~ n = 0 \rightarrow n, f : [n \ge 0 \land n = 0 \land n = N, f = n_0!]}
\gcl{0}{\Choice ~ n > 0 \rightarrow n, f : [n \ge 0 \land n > 0 \land n = N, f = n_0!]}
\gcl{0}{\Fi}
\hint{\refsto}{Assignment: $ n \ge 0 \land n = 0 \land n = N \entails (f = n_0!)[f \backslash 1] $}
\gcl{0}{\If ~ n = 0 \rightarrow f := 1}
\gcl{0}{\Choice ~ n > 0 \rightarrow n, f : [n \ge 0 \land n > 0 \land n = N, f = n_0!]}
\gcl{0}{\Fi}

The remaining specification $ n, f : [n \ge 0 \land n > 0 \land n = N, f = n_0!] $ is refined as follows:
\form{n, f : [n \ge 0 \land n > 0 \land n = N, f = n_0!]}
\hint{\refsto}{Contract frame: $ n $}
\form{f : [n \ge 0 \land n > 0 \land n = N, f = n!]}
\hint{\refsto}{Following assignment: $ f := f \times n $}
\gcl{0}{f : [n \ge 0 \land n > 0 \land n = N, (f = n!)[f \backslash f \times n]]; f := f \times n}

The remaining specification $ f : [n \ge 0 \land n > 0 \land n = N, (f \times n) = n!] $ is refined into a recursive call:\\
\form{f : [n \ge 0 \land n > 0 \land n = N, (f \times n) = n!]}
\hint{\refsto}{Apply substitution}
\form{f : [n \ge 0 \land n > 0 \land n = N, (f \times n) = n!]}
\hint{\refsto}{Divide both sides of $ f \times n = n! $ by $ n $}
\form{f : [n \ge 0 \land n > 0 \land n = N, f = (n - 1)!]}
\hint{\refsto}{Weaken precondition: $ n \ge 0 \land n > 0 \land n = N \entails n - 1 \ge 0 \land (0 \le n - 1 < N) $}
\form{f : [n - 1 \ge 0 \land (0 \le n - 1 < N), f = (n - 1)!]}
\hint{\refsto}{Apply substitution backwards}
\form{f : [(n \ge 0 \land (0 \le n < N))[n \backslash n - 1], (f = n!)[n_0, f \backslash n_0 - 1, f]]}
\hint{\refsto}{Introduce recursive call with $ \Value $ parameter $ n $ and $ \Result $ parameter $ f $}
\form{Factorial(n - 1, f)}

This produces the final program:\\
\gcl{0}{\If ~ n = 0 \rightarrow f := 1}
\gcl{0}{\Choice ~ n > 0 \rightarrow Factorial(n - 1, f); f := f \times n}
\gcl{0}{\Fi}