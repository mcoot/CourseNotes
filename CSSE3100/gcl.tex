\section{Guarded Command Language}

\begin{itemize}
	
	\item $ \Skip $ -- Empty Command
	
	\begin{itemize}
		
		\item $ wp(\Skip, Q) \equiv Q $
		
		\item Hence $ \hoare{Q}{\Skip}{Q} \equiv \true $ for any $ Q $
		
		\item $ \hoare{P}{\Skip}{Q} $ is false if and only if $ P $ is strictly weaker than $ Q $
		
	\end{itemize}

	\item $ \Abort $ -- Chaotic Command
	
	\begin{itemize}
		
		\item $ wp(\Abort, Q) \equiv \false $
		
		\item No precondition can guarantee a postcondition
		
		\item Represents `chaotic/undefined behaviour'
		
	\end{itemize}
	
	\item $ := $ -- Assignment
	
	\begin{itemize}
		
		\item $ wp(x := E, Q) \equiv Q[x \backslash E] $
		
		\item So $ \hoare{Q[x \backslash E]}{x := E}{Q} $ is $ \true $
		
		\item Generalises to multiple assignment: $ wp((x,y := E, F), Q) \equiv Q[x,y \backslash E,F] $
		
	\end{itemize}
	
	\item $ ; $ -- Composition/Concatenation
	
	\begin{itemize}
		
		\item $ wp(S_1; S_2, Q) \equiv wp(S_1, wp(S_2, Q)) $
		
		\item There exists some `middle' predicate true after $ S_1 $ and before $ S_2 $
		
		\item If $ \hoare{P}{S_1}{M} \land \hoare{M}{S_2}{Q} $ then $ \hoare{P}{S_1; S_2}{Q} $
		
	\end{itemize}
	
	\item $ \If $ -- Selection
	
	\begin{itemize}
		
		\item
		$ \If ~ G_1 \rightarrow S_1  $\\
		$ \Choice ~ G_2 \rightarrow S_2 $\\
		$ \dots $\\
		$ \Choice ~ G_n \rightarrow S_n $\\
		$ \Fi $
		
		\item Evaluate all guards $ G_1 \dots G_n$, choose a $ \true $ guard $ G_i $ nondeterministically, and execute $ S_i $
		
		\item if all guards evaluate to $ \false $ then $ \Abort $ is executed
	
		\item $ wp(\If, Q) \equiv \bigvee_{i=1}^{n} G_i \land \bigwedge_{i=1}^{n} (G_i \implies wp(S_i), Q)  $
		\item The disjunction of the guards \textit{must} be true
			
	\end{itemize}
	
	\item $ \Do $ -- Repetition
	
	\begin{itemize}
		
		\item
		$ \Do ~ G_1 \rightarrow S_1  $\\
		$ \Choice ~ G_2 \rightarrow S_2 $\\
		$ \dots $\\
		$ \Choice ~ G_n \rightarrow S_n $\\
		$ \Od $
		
		\item Evaluate all guards, choose a $ \true $ guard nondeterministically, execute $ S_i $ and repeat
		
		\item If all $ G_i $ are false, the loop ends and the program continues
		
		\item Weakest precondition rule is complex, so \textit{loop invariants} are used instead
		
		\item Predicate $ I $ is a loop invariant if $\hoare{I \land G_i}{S_i}{I} $ for all $ 1 \le i \le n $
		
		\item There are usually many possible invariants for a loop
		
	\end{itemize}
	
\end{itemize}